\hypertarget{index_intro}{}\section{Introduction}\label{index_intro}
This is a C++ header only library devoted to numerically computing the $RO(G)$ homology of a point. You can find the Git\+Hub repository \href{https://github.com/NickG-Math/Mackey}{\tt here}.

For a quick demonstration in the case of $G=C_4$ you can use one of the available binaries \href{https://github.com/NickG-Math/Mackey/tree/master/bin}{\tt here}.\hypertarget{index_req}{}\section{Requirements}\label{index_req}

\begin{DoxyItemize}
\item C++17 and the standard library.
\item \href{ http://eigen.tuxfamily.org/index.php?title=Main_Page}{\tt Eigen}, a header only library for matrix manipulation. I\textquotesingle{}ve tested the code with \hyperlink{namespaceEigen}{Eigen} 3.\+3.\+7 though newer versions shouldn\textquotesingle{}t break compatibility.
\item Optional\+: For improved performance you can use the Intel M\+KL with \hyperlink{namespaceEigen}{Eigen} and further combine with Open\+MP for multithreading.
\item Optional\+: To draw the multiplication graphs you will need Graphviz.
\end{DoxyItemize}\hypertarget{index_install}{}\section{Installation}\label{index_install}

\begin{DoxyItemize}
\item To install simply clone/download the \href{https://github.com/NickG-Math/Mackey}{\tt repository} and include it in your path. You will also need to do the same with \hyperlink{namespaceEigen}{Eigen}.
\item See the page \hyperlink{use}{How to Use} for a tutorial on using the library.
\item As for compiler support, I have tested the code with the following C++ compilers\+: G\+CC 9.\+2 (Linux), Clang 10 (Linux and Mac\+OS), Intel Compiler 19 (Linux and Windows), M\+S\+VC 19 (Windows). Remember to use the option {\ttfamily -\/std=c++17}. For more information on compiler options, see the \hyperlink{perf}{Performance} page.
\end{DoxyItemize}\hypertarget{index_status}{}\section{Current Status}\label{index_status}

\begin{DoxyItemize}
\item The project is almost complete for $G$ a cyclic group of prime power order. The only input that\textquotesingle{}s needed are the equivariant chains at the bottom level for the spheres corresponding to actual representations; we call these \char`\"{}standard chains\char`\"{}. The standard chains can be easily computed from geometric equivariant decompositions by hand, and then fed into the program as explained in \hyperlink{use_how}{Step 0\+: Setting the Group Parameters}. We have automated this process for prime 2.
\item The one thing that hasn\textquotesingle{}t been implemented for prime-\/power cyclic groups are Frobenius relations\+: The multiplicative structure is computed levelwise, but this could be made more effective using the Frobenius relations.
\item For general cyclic groups a few aspects that involve transferring need reworking. The problem is that non prime-\/power cyclic groups the diagram of subgroups is not a vertical tower but a somewhat more complicated diagram, so care has to be taken to account for all these extra transfers and restrictions. Ultimately this is the only part that needs changing.
\item The bulletpoint above also applies to general finite abelian groups. We also need to specify the order of the elements of the group and how they relate to the subgroup diagram to form our equivariant bases.
\item For non abelian groups we have the added complication of needing the real representation theory of our group. And of course we need the standard chains for these groups as well.
\item For non constant coefficients a lot more things start to break, as transferring becomes more complicated when non cyclic modules are involved in the free \hyperlink{namespaceMackey}{Mackey} functors.
\end{DoxyItemize}\hypertarget{index_doc}{}\section{Documentation}\label{index_doc}
This documentation is organized in pages as follows\+:


\begin{DoxyItemize}
\item The \href{namespaces.html}{\tt Related pages} consist of the pages \hyperlink{index}{General Information}, \hyperlink{math}{From Math to Code}, \hyperlink{use}{How to Use}, \hyperlink{algo}{Algorithm Details}, \hyperlink{perf}{Performance}. These explain how the program works, starting from the math and moving to slightly more technical territory regarding the actual implementation.
\item The pages \href{namespaces.html}{\tt Namespaces}, \href{annotated.html}{\tt Classes} and \href{files.html}{\tt Files} are automatically generated by doxygen from the source code (and comments in the source code). These offer a much more indepth look into all classes and functions of this project. Note that only public and protected members and named namespaces are documented.
\item I recommend starting with the \href{namespaces.html}{\tt related pages} before moving to the automatically generated ones. If you just want to use this library for computations, you only really have to go over the \hyperlink{use_how}{Step 0\+: Setting the Group Parameters} section. 
\end{DoxyItemize}