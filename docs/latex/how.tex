\hypertarget{how_para}{}\section{The parameters}\label{how_para}

\begin{DoxyItemize}
\item For every group there are certain mandatory parameters that need to be set for the library to work. We will go over them in more detail below.
\item There are also some optional ones that allow you to identify and print the names of the computed \hyperlink{namespaceMackey}{Mackey} functors. This functionality is disabled by default, but can easily be turned on by defining the macro {\ttfamily M\+A\+C\+K\+E\+Y\+\_\+\+N\+A\+M\+ES} and setting the optional parameters. The way this is done is explained in the file Optional\+\_\+\+Implementation.\+h in the Demo folder here ... . The implementation there is for $G=C_4$ and the \hyperlink{namespaceMackey}{Mackey} functors in the $RO(C_4)$ homology.
\end{DoxyItemize}\hypertarget{how_manda}{}\section{The mandatory parameters}\label{how_manda}
We have included an example (for $G=C_4$) of how to set the group parameters in Implementation.\+h of the Demo folder (...). These parameters must be set before {\ttfamily main} starts. They all live in the \hyperlink{namespaceGroupSpecific}{Group\+Specific} namespace.\hypertarget{how_var}{}\subsection{Global variables}\label{how_var}
The global variables that need to be set are\+:


\begin{DoxyItemize}
\item \hyperlink{classGroupSpecific_1_1Variables_a38586ec998bcbfdf325e6cfc6598b54b}{prime} \+: the $p$ in $G=C_{p^n}$.
\item \hyperlink{classGroupSpecific_1_1Variables_ac9bd6be19cc41e6877ee25a2d1c7be80}{power}\+: the $n$ in $G=C_{p^n}$.
\item \hyperlink{classGroupSpecific_1_1Variables_a5504789b0b60050e6ea223fdeb84874a}{reps} \+: the number of nontrivial irreducible real representations of $G$.
\item \hyperlink{classGroupSpecific_1_1Variables_a4746f16736abcf4c705dd8690ec12ca0}{sphere\+\_\+dimensions} \+: the array consisting of the dimensions of those representations (so we must fix an order for them beforehand).
\end{DoxyItemize}\hypertarget{how_fun}{}\subsection{The standard differentials}\label{how_fun}

\begin{DoxyItemize}
\item There is one function that needs to be manually defined, and that\textquotesingle{}s \hyperlink{classGroupSpecific_1_1Function_a8ead55e2f2e2bbda4deea3964793498d}{Standard\+Diff} creating the differential for the standard chains. In practice what this means is setting the correct matrix given the input sphere. Apart from the case work that comes from math, I have made the construction of the differential as painless as possible.
\item This construction is done through the \hyperlink{namespaceMackey_a26a529f63caac9c5b4dc809e0e5831be}{altmatrix} function, that creates alternating matrices of the desired size and the desired \char`\"{}pattern\char`\"{}. This pattern is repeated cyclically in the columns of the matrix.
\item An example\+: The matrix of size 4x4 with pattern $a,b$ is ~\newline
 $\begin{matrix} a&b&a&b\\ b&a&b&a\\ a&b&a&b \\ b&a&b&a \end{matrix}$ ~\newline
 If we use the pattern $a,b,c,d$ instead we get ~\newline
 $\begin{matrix} a&b&c&d\\ b&c&d&a\\ c&d&a&b \\ d&a&b&c \end{matrix}$ ~\newline

\item The matrices appearing in the Standard Chains all look like this due to equivariance. 
\end{DoxyItemize}